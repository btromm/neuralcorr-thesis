%*******************************************************
% Abstract
%*******************************************************
%\renewcommand{\abstractname}{Abstract}
\pdfbookmark[1]{Abstract}{Abstract}
% \addcontentsline{toc}{chapter}{\tocEntry{Abstract}}
\begingroup
\let\clearpage\relax
\let\cleardoublepage\relax
\let\cleardoublepage\relax

\chapter*{Abstract}
Neurons are capable of maintaining consistent activity over a long time scale, while undergoing perturbations from factors such as ion channel turnover and activity perturbations. Activity-dependent feedback has been proposed as a homeostatic regulatory mechanism that can regulate channel densities according to the firing rate activity of the neuron. Homeostatic regulation rules can impose correlations on the mRNA- and maximal conductance-level steady state distribution of ion channels. However, it is not currently known how these tuning rules result in the emergence of ion channel correlations. This thesis explores how variability in the parameters of homeostatic tuning rules, such as intracellular Ca\textsuperscript{2+} concentration targets, transcription rates, and translation rates can result in the emergence of diverse, robust correlations between ion channels in neurons.
\vfill

\endgroup

\vfill
