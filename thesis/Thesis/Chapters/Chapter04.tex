%************************************************
\chapter{Conclusion}\label{ch:conclusion}
%************************************************

Neurons are capable of processing and integrating an enormous variety of information, remaining sensitive to internal cues and modulatory biochemical and electrical inputs, while being robust to environmental perturbations and ion channel turnover.
Neurons can maintain consistent patterns of activity despite variability within behaviorally equivalent types of neurons through activity-dependent feedback, a homeostatic regulatory mechanism which regulates channel mRNA levels and maximal conductances dependent on the firing rate activity of the neuron\cite{lemasson_activity-dependent_1993,liu_model_1998,stemmler_how_1999}. Activity-dependent feedback has been shown experimentally to produce compensatory changes in ion channel maximal conductance densities and mRNA concentrations after prolonged changes in firing rate activity. Furthermore, activity-dependent feedback has been shown to widen the range of characteristic sets of mRNA levels and conductance densities that can produce equivalent types of firing rate activity\cite{rodgers_dopaminergic_2013,goldman_global_2001}.

Most important to this thesis, however, is how activity-dependent feedback as a process of homeostatic regulation can produce diverse, highly variable correlations between ion channels\cite{schulz_variable_2006,schulz_quantitative_2007,golowasch_activity-dependent_1999}.  These correlations are ubiquitous across phyla, are genetically conserved, and are well documented across species and within cell types\cite{tran_ionic_2019,tobin_correlations_2009,schulz_quantitative_2007,santin_membrane_2019,maclean_activity-independent_2003}.
These correlations are important due to their ability to make neurons robust to environmental perturbations and ion channel turnover.
Changes in maximal conductances of one ion channel can result in compensatory changes in others to preserve the activity of the neuron, indicating that these correlations allow for neurons to co-regulate ion channels -- reducing the complexity of maintaining functional output within the cell.
Co-regulation of ion channels makes neurons robust to perturbation by providing a mechanism by which neurons can up- or down-regulate ion channels dependent on the activity of its correlated channels.

Despite significant experimental observations across species and cell types, it is unknown how properties of ion channel correlations such as magnitude and connection strengths can emerge from homeostatic regulation.
In this thesis, we explored how an activity-dependent feedback model based on the integral control scheme established by O'Leary \textit{et al.} produces variability in magnitude and connection strengths of correlations between ion channels\cite{oleary_correlations_2013}.
A systematic examination of the different intrinsic properties of neurons and the homeostatic regulatory mechanism was carried out.
We first examined the magnitude of variability and strength of correlations between steady state maximal conductances after variation in initial channel maximal conductance densities and mRNA levels. It was found that the model was highly insensitive to variability in initial conditions, compressed variability in steady state channel maximal conductances by a significant factor, and was unable to reproduce either the magnitude of variability between steady state channel maximal conductances or the strength of correlation between maximal conductances.
We then examined the impact of variability in \ac{gleak} on these same properties of correlations between steady state channel maximal conductances. It was found that variability in the unregulated passive ion leak conductance was sufficient to reproduce a linear pairwise correlation between steady state maximal conductances, but the magnitude of variability was less than two-fold\cite{golowasch_activity-dependent_1999}. Along with that, all steady state maximal conductances were perfectly correlated.

It was thus determined that variability in magnitude and strength of correlations between ion channel maximal conductances was not significantly influenced by intrinsic properties of the neuron. Rather, variability in properties of the homeostatic regulatory mechanism significantly influences the magnitude of variability between maximal conductances and the strength of correlation.
It was first shown that variability in \ac{Ca_target} was sufficient to reproduce the magnitude of variability between steady state maximal conductances observed in experimental data. However, correlations between steady state maximal conductances were perfectly correlated, which is inconsistent with experimental observations which show varied connection strengths between ion channels\cite{santin_membrane_2019}.
Finally, the transcription and translation rate constants were varied to explore if these had any influence on the patterns of correlations between ion channels. It was unsurprising to find that variability in \ac{taug} alone was insufficient to reproduce either the magnitude or connection strengths of correlations, given that the rate of translation into ion channel protein is dependent on the rate of mRNA transcription and concentration of channel mRNAs.

Variability in \ac{taum} was sufficient to reproduce the experimentally observed magnitude in variation within ion channel maximal conductance distributions and strengths of correlation between ion channel maximal conductances.
While in all other simulations, if linear pairwise correlations were produced at all, channel maximal conductances were perfectly correlated. In contrast, varying \ac{taum} produced correlations where changes in one channel could not be fully explained by changes in another. Variability in rates of transcription, but not any other parameters, reproduced the full range of properties of correlations found in experimental observations\cite{santin_membrane_2019,schulz_quantitative_2007,schulz_variable_2006}. Thus, it has been shown that robust and diverse correlations can emerge from homeostatic tuning rules and variability in the rates of transcription of ion channel mRNAs.

